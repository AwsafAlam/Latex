\documentclass{beamer}
\usepackage[utf8]{inputenc}
\usepackage{subfiles}
\usepackage{lipsum}
\usepackage{float}
\usepackage{listings}
\usepackage{color}
\usepackage{tikz}
\usepackage{lmodern}

\definecolor{darkblue}{rgb}{0.086,0.145,0.357}
\definecolor{darkpurple}{rgb}{0.3098,0 , 0.3098}
\definecolor{skinorange}{rgb}{0.957,0.639 , 0.365}


\usetikzlibrary{arrows.meta,arrows}
\usetikzlibrary{positioning}

\usetheme{Madrid}
\usecolortheme{beaver}

\title[Priority Queue]{Priority Queue}
\subtitle{An Introduction}

\author[A.A. and A.N.F]{Md Awsaf Alam \inst{1} \\ Ahmed Nafis Fuad\inst{2}}

\institute{
\inst{1}
Department of CSE\\
BUET\\
\inst{2}
Department of CSE\\
BUET
}

\date{\today}

\AtBeginSection[]{
\begin{frame}
    \tableofcontents[currentsection]
\end{frame}
}

\begin{document}
\titlepage

\begin{frame}{Table of Contents}
\tableofcontents

\end{frame}

\section{What is a Priority Queue?}
\subfile{sec/definition.tex}


\section{Application of Priority Queue}
\subfile{sec/application.tex}

\section{Other Applications}
\begin{frame}{Simulation}
  A Binary (Max) Heap is a complete binary tree that maintains the Max Heap property.
  Binary Heap is one possible data structure to model an efficient Priority Queue (PQ) Abstract Data Type (ADT). In a PQ, each element has a "priority" and an element with higher priority is served before an element with lower priority (ties are broken with standard First-In First-Out (FIFO) rule as with normal Queue). Try clicking ExtractMax() for a sample animation on extracting the max value of random Binary Heap above.
  To focus the discussion scope, we design this visualization to show a Binary Max Heap that contains distinct integers only.
\end{frame}

\section{Implementations of Priority Queue}
\begin{frame}{Implementations of Priority Queue}
\setbeamercovered{dynamic}
\begin{itemize}
  \item Fibonacci Heap
  \item Binary Heap
\end{itemize}

\end{frame}


\section{Previous Works}
\begin{frame}{Blocks}
    \begin{block}{Sample Block}
    This is a sample block.
    \end{block}
    \begin{alertblock}{Sample Alert Block}
    This is a sample \textbf<2>{alert block}.
    \end{alertblock}
    \begin{example}
    Sample \textcolor<3->{red}{example}.
    \end{example}
\end{frame}

\section{Binary Max Heap}
\begin{frame}{Definition}
  A Binary (Max) Heap is a complete binary tree that maintains the Max Heap property.
  Binary Heap is one possible data structure to model an efficient Priority Queue (PQ) Abstract Data Type (ADT). In a PQ, each element has a "priority" and an element with higher priority is served before an element with lower priority (ties are broken with standard First-In First-Out (FIFO) rule as with normal Queue). Try clicking ExtractMax() for a sample animation on extracting the max value of random Binary Heap above.
  To focus the discussion scope, we design this visualization to show a Binary Max Heap that contains distinct integers only.
\end{frame}
\subfile{sec/tree_animation.tex}

\section{Conclusion}
\begin{frame}{The End}
    Any Questions?
\end{frame}
\end{document}
