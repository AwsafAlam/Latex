\documentclass{book}

\usepackage{lipsum}
\usepackage{color}
\usepackage{amsmath}
\usepackage{graphicx}

\title{Microcontrollers \& Embeded System Design ~\textbf{\\CSE 315}}
\author{Md Awsaf Alam}
\date{\today}

\begin{document}

\maketitle
% \lipsum[1-3]
\newpage

\tableofcontents

\chapter{8086/8088 Hardware Specifications}

\section{Class Lec 2}

\subsection{Introductory text}
This is the \textbf{first} CSE 300 class in Jan 2018 term.
This is a new \textcolor{blue}{\textbf{line}}.
This is a new line.
This is a new line.
This is a new line.
This is a new line.


\subsection{Introductory text}
\label{sec:intro_text}
This is the \textbf{first} CSE 300 class in Jan 2018 term.
This is a new \textcolor{red}{\textbf{line}}.
This is a new line.
This is a new line.

\subsection{description}

\begin{description}
    \item[\overline{RD}]
    \begin{itemize}
        \item  Whenever this pin goes to logic 0, the data bus becomes receptive to data from the memory or I?O devices
            connected to the system.
        \item Floats to high impedence state during a hold acknowledge
    \end{itemize}

    \item[READY]
    \begin{itemize}
        \item uP enters into WAIT state and remains idle if this pin is at logic 0
        \item No effect on operations of up, if this pin is at logic 1
    \end{itemize}


    \item[INTR]
    \begin{itemize}
        \item  Used to request a h/w interrupt
        \item  If INTR is held high when $IF = 1$, the up enters an interrupt acknowledge cycle
        ( \textcolor{purple}{INTA} becomes active) after completion of the current instruction
    \end{itemize}

    \item[TEST]
    \begin{itemize}
        \item  An input that is tested by the WAIT instruction
        \item  If TEST is logic 0, the \textcolor{red}{WAIT} instruction functions as \textcolor{red}{NOP}
        \item If TEST is logic 1, the \textcolor{red}{WAIT} instruction waits for \textit{blue}{TEST} to become 0


   \end{itemize}

   \item[NMI]
   \begin{itemize}
       \item Non markable interrupt pin
       \item Similar to the INTR except that NMI does not check IF (whether it is 1)

   \end{itemize}

   \item[RESET]
   \begin{itemize}
       \item Causes the uP to reset itself if this pin remains high for a minimum of four clocking periods
      \item whenever the up gets reset , it begins executing instructions at memory location FFFFOH
      and disables future interrupts by clearing IF
   \end{itemize}

   \item[CLK]
   \begin{itemize}
       \item  Provides the base timing signal to the up
       \item Clock signal must have at least 33\% duty cycle (high for the one-third
       of the clocking period and low for two-third of the period)

  \end{itemize}

  \item[VCC]
  \begin{itemize}
      \item Power supply input
      \item Provides $+5.0$ volt with 10\% tolerance to the up

  \end{itemize}

  \item[GND]
  \begin{itemize}
      \item 2 pins, both must be connected to ground

  \end{itemize}

  \item[MN/MX]
  \begin{itemize}
      \item Selects either minimum mode or maximum mode operation of the up
  \end{itemize}


  \item[BHE/S7]
  \begin{itemize}
      \item Bus high Enable
      \item Used in 8086 to enable the most signifant data bus bits (D15 - D8) during a read or
      write operations
      \item The state of S7 is always a logic 1
  \end{itemize}

% \subsection{Minimum Mode Pins}

Minimum Mode Pins

  \item[IO/M or M/IO]
  \begin{itemize}
      \item Selects memory or I/O
      \item Indicates
  \end{itemize}


\end{description}



\section{Figures}

Here, we will import figures. Figure \ref{fig:bl} is our figure.

\begin{figure}[h!]
    \centering
    \includegraphics[width = 0.3\textwidth]{figures/logo.png}
    \caption{BUET Logo}
    \label{fig:bl}
\end{figure}

\lipsum

\end{document}
