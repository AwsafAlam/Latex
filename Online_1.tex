\documentclass{article}
\usepackage[utf8]{inputenc}

\usepackage{lipsum}
\usepackage{color}
\usepackage{amsmath}
\usepackage{graphicx}

\newcommand{\FT}{Fourier Transform}
\newcommand{\anothertestcommand}[2]{Welcome, #2! Mighty #1!!!}

\title{CSE300\_Online1\\Differential Equation}
\author{Md Awsaf Alam\\Student ID: 1505114}
\date{\today}

\begin{document}

\maketitle

\newpage

\section{Existence of solutions}
Solving differential equations is not like solving \textcolor{blue}{algebraic equations}. Not only are their solutions often unclear, but whether solutions are unique or exist at all are also notable subjects of interest.

\par For first order initial value problems, the \textcolor{blue}{Peano existence theorem} gives one set of circumstances in which a solution exists. Given any point $\textbf{\textit{(a,b)}} $ in the xy-plane, define some rectangular region {\displaystyle Z} , such that {\displaystyle Z=[l,m]\times [n,p]} and  $\textbf{\textit{(a,b)}} $ is in the interior of . If we are given a differential equation $\frac {dy}{dx} = g(x,y) $ and the condition that $ y=b $when $x=a$, then there is locally a solution to this problem if $ g(x,y)$ and $\frac {\partial g}{\partial x}$ are both continuous on {\displaystyle Z} . This solution exists on some interval with its center at  \textbf{\textit{a}}. The solution may not be unique. (See \textcolor{blue}{Ordinary differential equation} for other results.)

\par However, this only helps us with first order initial value problems. Suppose we had a linear initial value problem of the nth order:
\begin{equation}
    f_n(x) \frac{d^n y}{dx^n} + \cdot \cdot \cdot+f_1(x)\frac{dy}{dx} + f_0(x)y = g(x)
\end{equation}

\section{Figure}
\begin{figure}[h!]
    \centering
    \includegraphics[width = 0.3\textwidth]{figures/sketch.png}
    \caption{Visualization of heat transfer in a pump casing, created by solving the heat equation. Heat is being generated internally in the casing and being cooled at the boundary, providing a steady state temperature distribution.}
    \label{fig:bl}
\end{figure}


\end{document}
